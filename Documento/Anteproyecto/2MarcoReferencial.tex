\section{Marco referencial.}

\subsection{Marco te\'orico.}

\textbf{Procrastinación.} Es el hábito de posponer tareas o responsabilidades cruciales en favor de actividades más inmediatas o gratificantes, aunque menos relevantes a largo plazo. Esta tendencia puede estar influenciada por una variedad de factores psicológicos y ambientales, como la ansiedad ante el fracaso, la falta de motivación intrínseca para completar las tareas asignadas y la presencia constante de distracciones, entre las cuales destaca el uso excesivo del smartphone. La combinación de estos elementos puede llevar a un ciclo perjudicial de aplazamiento continuo, afectando negativamente tanto el rendimiento académico como el bienestar emocional de los estudiantes \cite{Beutel2016}.

\textbf{Tecnología y adicción.} La creciente adicción a la tecnología, particularmente a los dispositivos móviles y sus aplicaciones, ha captado la atención de investigadores y profesionales en salud mental. Estos dispositivos están diseñados con características específicas destinadas a maximizar la participación del usuario, desde las notificaciones constantes hasta la gamificación de las experiencias de usuario. Estas estrategias pueden desencadenar comportamientos adictivos al generar una gratificación instantánea y dificultar la capacidad de autorregulación del tiempo de uso. Como resultado, los usuarios pueden encontrarse atrapados en un ciclo de consumo compulsivo de contenido digital, lo que puede tener repercusiones negativas en su bienestar psicológico y en su capacidad para concentrarse en actividades importantes, como el estudio académico \cite{Montag2019}.

\textbf{Teoría del diseño centrado en el usuario.} Es una metodología orientada a crear productos y servicios que se adapten de manera óptima a las necesidades, preferencias y habilidades de los usuarios finales. El diseño centrado en el usuario implica iteraciones continuas y pruebas de usabilidad para garantizar que el producto final sea fácil de usar y cumpla con las expectativas de los usuarios. Esto implica la creación de prototipos y la realización de pruebas con estudiantes para evaluar la funcionalidad, la accesibilidad y la eficacia del launcher en la mejora de la gestión del tiempo y la reducción de la procrastinación.

\textbf{Psicología del color.} La Psicología del Color resalta cómo los colores influyen en las emociones y el comportamiento humano. En el diseño del launcher, la elección cuidadosa de colores es crucial, ya que puede impactar la motivación, enfoque y compromiso del usuario con las tareas académicas. Los tonos cálidos como el rojo y el naranja pueden estimular la energía y la acción, fomentando la productividad. Mientras tanto, colores suaves como el azul y el verde pueden crear un ambiente tranquilo, ideal para momentos de concentración y estudio. Mantener consistencia en la paleta de colores y su integración con la interfaz del launcher puede mejorar la experiencia del usuario, aumentando su disposición a utilizar la aplicación de manera efectiva para gestionar su tiempo y combatir la procrastinación.

\textbf{Gestión del tiempo.} Es el proceso de planificar, organizar y controlar cómo se utiliza el tiempo disponible para lograr objetivos específicos, tanto personales como profesionales. Implica identificar las tareas prioritarias, asignarles el tiempo adecuado y utilizar técnicas y herramientas para maximizar la eficiencia y la productividad. El launcher podría ofrecer funciones como recordatorios personalizados para tareas importantes, integración de calendario para una visualización clara de horarios y compromisos, y la capacidad de establecer objetivos académicos con seguimiento de progreso. Estas características permitirían a los estudiantes planificar y organizar sus actividades de manera efectiva, evitando conflictos y garantizando el cumplimiento de plazos, mientras identifican áreas de mejora para optimizar su rendimiento académico \cite{Mengual2012}. 

\textbf{Productividad.} Es la capacidad de producir más resultados con la misma cantidad de recursos, o producir los mismos resultados con menos recursos, en un período de tiempo determinado. Esto es esencial para el éxito tanto académico como personal, implicando la eficiencia en la realización de tareas y la obtención de resultados satisfactorios. En el contexto del launcher, se busca potenciar la productividad de los estudiantes mediante la minimización de distracciones y el logro de objetivos de manera efectiva. Esto se logra a través de funciones que permiten enfocarse en tareas prioritarias y limitar las interrupciones, organizar aplicaciones de manera intuitiva para acceder rápidamente a recursos de estudio, y proporcionar herramientas de seguimiento del tiempo para analizar y mejorar la eficiencia en el uso del dispositivo. 


\subsection{Estado del arte.}

El estudio "Adicción a las Redes Sociales y Procrastinación Académica en estudiantes Universitarios" \cite{Luisa2019} encontró una correlación positiva y significativa entre la adicción a las redes sociales y la procrastinación académica, lo que significa que niveles más altos de adicción a las redes sociales corresponden a niveles más altos de procrastinación académica. El estudio también destaca la prevalencia de ambos problemas entre los estudiantes universitarios de todo el mundo, ya que sólo el 15\% de los estudiantes no muestra ningún nivel de adicción a las redes sociales. Los resultados se basan en una muestra de estudiantes de Lima y son relevantes para comprender el impacto de las redes sociales en el rendimiento académico. Este estudio proporciona evidencia de la relación significativa entre la adicción a las redes sociales y la procrastinación académica en estudiantes universitarios, lo cual es una preocupación creciente en la educación superior. Los hallazgos sugieren que abordar la adicción a las redes sociales puede ser una estrategia eficaz para reducir la procrastinación académica y mejorar los resultados de los estudiantes.

El artículo "La Generación Zombie. El uso excesivo de teléfonos celulares en las aulas universitarias peruanas" \cite{Montenegro2023}. Analiza el uso excesivo de celulares por parte de estudiantes universitarios en las aulas de clase y sus efectos negativos en el aprendizaje y la salud. Se destaca que dos tercios de los estudiantes encuestados reconocen que el uso del celular en el aula afecta negativamente su aprendizaje al distraerlos y alejarlos de prestar atención a las actividades académicas. De igual manera resalta que casi el 70\% de los estudiantes está consciente de que el uso excesivo del celular es nocivo para la salud física y mental, generando adicción, problemas de concentración, afectando la interacción humana, etc. Por último, sugieren que las universidades deberían implementar políticas y estrategias para regular el uso de teléfonos móviles en las aulas, como horarios designados para el uso del teléfono y promover el uso responsable.

Adiba Orzikulova destaca los impactos negativos del uso excesivo de teléfonos inteligentes, particularmente en aplicaciones de redes sociales, en los estudiantes universitarios \cite{Orzikulova2022}. El estudio sugiere que las intervenciones de autoayuda basadas en aplicaciones móviles pueden ser efectivas para prevenir el uso excesivo de teléfonos inteligentes y la adicción a los teléfonos inteligentes, pero se necesita más investigación para comprender los mecanismos subyacentes de la adicción a los teléfonos inteligentes y el impacto de estas intervenciones en el comportamiento de los estudiantes. El estudio también enfatiza la importancia de la autorregulación y sugiere que las barreras físicas pueden ser más efectivas para prevenir la adicción a los teléfonos inteligentes y promover la autorregulación que las intervenciones basadas únicamente en software.


\subsection{Marco conceptual.}

\textbf{Aprendizaje autodirigido.} Es un proceso de aprendizaje en el que el estudiante lleva las riendas de su propio proceso educativo. Implica que el alumno identifica de forma autónoma sus necesidades y objetivos de aprendizaje, busca y selecciona los recursos y estrategias más adecuados, y evalúa por sí mismo sus logros y resultados. Requiere que el estudiante emplee habilidades de autorregulación, como estrategias cognitivas, metacognitivas y de motivación personal, ya sea trabajando de manera independiente o con cierta guía externa. En esencia, el aprendiz toma un papel activo y autorresponsable en la conducción de su aprendizaje \cite{Marquez2014}.

\textbf{Launcher Android.} Un launcher en Android es una aplicación que permite personalizar la interfaz de usuario y la apariencia de la pantalla de inicio de un dispositivo móvil. Funciona como una capa de personalización sobre el sistema operativo Android, permitiendo al usuario modificar la disposición de iconos, widgets, fondos de pantalla y otros elementos visuales en la pantalla de inicio. Los launchers ofrecen opciones de personalización avanzadas, como cambiar el diseño de la pantalla de inicio, agregar efectos de transición, modificar los iconos de las aplicaciones y ajustar la organización de las aplicaciones \cite{Launcher}.

\textbf{Autocontrol.} Capacidad de modular y controlar las propias acciones de una forma apropiada a la edad de la persona. Es considerado uno de los componentes clave de la inteligencia emocional que debe ser reeducado en los estudiantes. El autocontrol implica tener una sensación de control interno sobre el propio cuerpo, conducta y entorno, permitiendo regular los impulsos y actuar de manera intencionada para lograr los objetivos deseados. Se plantea que el desarrollo del autocontrol desde la infancia constituye una facultad fundamental en el ser humano para tener una voluntad sólida y capacidad de autogobernarse \cite{Navarro2003}. 

\textbf{Rendimiento académico.} El rendimiento académico se refiere a la evaluación y medición de los conocimientos y habilidades adquiridos por un estudiante durante su formación académica, ya sea en niveles escolares, terciarios o universitarios. Se considera que un alumno tiene un buen rendimiento académico cuando obtiene calificaciones positivas y aprobatorias en los exámenes y evaluaciones que rinde a lo largo de los ciclos o períodos académicos correspondientes \cite{RendimientoEscolar}.

\textbf{Concentración.} La concentración es el enfoque voluntario de la mente hacia un objetivo específico, tarea o actividad, excluyendo cualquier distracción o interferencia que pueda surgir. Es un proceso psíquico que se lleva a cabo mediante el razonamiento, donde se dirige toda la atención hacia un punto determinado, ya sea una tarea en curso o algo en lo que se está pensando \cite{Concentracion2023}.

\textbf{Frustración.} La frustración es un estado emocional caracterizado por sentimientos de decepción, irritabilidad o insatisfacción que surgen cuando una persona experimenta obstáculos o dificultades para alcanzar sus metas o satisfacer sus necesidades. Se manifiesta como una respuesta emocional negativa frente a la percepción de que los esfuerzos realizados no han dado los resultados deseados. La frustración puede surgir en diversas situaciones de la vida cotidiana, como problemas laborales, académicos, personales o sociales, y puede tener efectos adversos en el bienestar emocional y el comportamiento de la persona afectada \cite{Kamenetzky2009}.