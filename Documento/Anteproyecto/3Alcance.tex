\section{Alcance.}	

\subsection{Declaración del alcance.}

El alcance del presente trabajo de grado consiste en el desarrollo de un launcher Android personalizado para estudiantes de la Universidad del Valle en Tuluá que incorpore las siguientes funcionalidades: 

\begin{enumerate}
    \item \textbf{Diseño minimalista:} Contará con una interfaz sin íconos de aplicaciones para evitar desviar la atención, en tonos neutrales y con atajos útiles para gestionar las demás características presentes en el launcher.
    \item \textbf{Seguimiento del tiempo de uso de las aplicaciones:} El tiempo que se pasa en ciertas aplicaciones seleccionadas por el estudiante se podrá regular y monitorear con el fin de registrar progreso o áreas de mejora.
    \item \textbf{Gestión de tareas y hábitos:} El launcher permitirá consignar las tareas que se deseen realizar y crear hábitos en función de tareas, tiempo límite o chequeo regular del hábito en cuestión.
    \item \textbf{Herramienta(s) de productividad:} Se integrará al menos una herramienta de productividad para apoyar la realización de las tareas o los hábitos, como técnicas de distribución del tiempo de sesiones de trabajo, priorización de tareas, entre otras.
\end{enumerate}

%%%%%%%%%%%%%%%%%%%%%%%%%%%%%%%%%%%%%%%%%%%%

\subsection{Supuestos.}

\begin{enumerate}
    \item Correcto funcionamiento de los equipos donde se llevará a cabo el desarrollo y las pruebas del proyecto. 
    \item Normal desarrollo de los periodos académicos.
    \item Continuidad laboral del director de trabajo de grado.
\end{enumerate}

%%%%%%%%%%%%%%%%%%%%%%%%%%%%%%%%%%%%%%%%%%%%

\subsection{Restricciones.}

\begin{enumerate}
    \item El proyecto se debe llevar a cabo en un plazo máximo de ocho (8) meses.
    \item Solo estará disponible para el sistema operativo Android.
    \item Será diseñado para el tamaño y la interfaz de un smartphone, no para tablets u otros dispositivos similares.
    \item El proyecto y anteproyecto deben ser aprobados.
\end{enumerate}

