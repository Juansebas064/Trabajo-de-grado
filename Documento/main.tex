% !TeX root = main.tex
\documentclass[letterpaper, 12pt, oneside]{article}

\usepackage[utf8]{inputenc} %Para configuración de caracteres
\usepackage[spanish]{babel} %Para configuración de idioma

\usepackage{float} %Para que funcionen las tablas
\usepackage{anysize} %Para usar márgenes
\marginsize{3cm}{2.5cm}{1cm}{2cm} %{izquierda}{derecha}{arriba}{abajo}. La superior esta sobre 1, la derecha sobre -0.5 y la de abajo sobre 2 (por el número de página)

%tablas
\usepackage{booktabs}
\usepackage{longtable}
%rotar tablas (y usar \includegraphics )
\usepackage{rotating}
%Dividir en diagonal
%\usepackage{slashbox}
%color tablas
\usepackage{colortbl}
%colocar tabla en lugar de cuadro

\usepackage[numbib,notlof,notlot,nottoc]{tocbibind} %Para que la biliografía se llame así y no referencias y para que quede numerada como sección.
%\usepackage[notlof,notlot,nottoc]{tocbibind} %Bibliografía sin numerar
\bibliographystyle{ieeetr} %Estilo de bibliografía IEEE

\usepackage{url} % inserción url's notas de pie.

%Para poder colocar texto en color
\usepackage{color}
\definecolor{naranja}{rgb}{1,0.5,0} % valores de las componentes roja, verde y azul (RGB)
\definecolor{rojo}{rgb}{1,0,0}
\definecolor{SteelBlue}{rgb}{0.3,0.5,0.7}

%Para tener los bookmarks del pdf (menú al lado derecho en los visualizadores de pdf)
% si colorlinks= true, no salen las cajas, sino el color del link!!
% linkcolor para los indices, citecolor para las citas en texto, urlcolor para los enlaces
\usepackage[pdftex,bookmarks=true, linkcolor=black, citecolor=black, colorlinks=true, urlcolor=black]{hyperref}

%Para poder hacer subfiguras (subfloat)
\usepackage{subfig}

\usepackage{enumitem} % Para poder continuar enumerate en otras partes

\usepackage{pdflscape}%Para colocar páginas horizontales en el PDF

\begin{document}
	
\renewcommand{\tablename}{Tabla}
%%%%%%%%%%%%%%%%%%%%%%%%%%%%%% Portada %%%%%%%%%%%%%%%%%%%%
\newcommand\portada{
\begin{titlepage}
		\begin{center}
			{\bf Diseño e implementación de un launcher Android enfocado a combatir la procrastinación por el uso excesivo del smartphone en estudiantes de pregrado de la Universidad del Valle}
			\vfill
			{\bf Juan Sebastián Ruiz Aguilar}
			\vfill
			{\bf Universidad del Valle\par}
			{\bf Facultad de ingeniería \par}
			{\bf Escuela de ingeniería de sistemas y computación \par}
			{\bf Tuluá, Valle del Cauca \par}
			{\bf 2025 \par}
		\end{center}
\end{titlepage}
}

\newcommand\contraportada{
	\begin{titlepage}
		\begin{center}
			{\bf Diseño e implementación de un launcher Android enfocado a combatir la procrastinación por el uso excesivo del smartphone en estudiantes de pregrado de la Universidad del Valle}
			\vfill
			\vfill
			\vfill
			{\bf Juan Sebastián Ruiz Aguilar \par}
			{\bf Código: 2059898 \par}
			{\url{juan.ruiz.aguilar@correounivalle.edu.co} \par}
			\vfill
			\vfill
			\vfill
			\vfill
			{Director \par}
			{\bf Ing. Héctor Fabio Ocampo Arbeláez \par}
			{Magister en analítica e inteligencia de negocios \par}
			{\url{hector.ocampo@correounivalle.edu.co} \par}
			\vfill
			\vfill
			\vfill
			{\bf Universidad del Valle\par}
			{\bf Facultad de ingeniería \par}
			{\bf Escuela de ingeniería de sistemas y computación \par}
			{\bf Tuluá, Valle del Cauca \par}
			{\bf 2025 \par}
		\end{center}
\end{titlepage}
}

% Ahora llama a los comandos para que se muestren:
\portada
\contraportada

%%%%%%%%%%%%%%%%%%%%%%%%%%%%%% Tabla de contenido %%%%%%%%%%%%%%%%%%%%
\renewcommand\contentsname{\centering Tabla de Contenido}
\tableofcontents

%%%%%%%%%%%%%%%%%%%%%%%%%%%%%% Resumen %%%%%%%%%%%%%%%%%%%%%%%%%%%%%%%
\begin{center}
	\section*{Resumen}	
\end{center}
\addcontentsline{toc}{section}{Resumen} %Añadir resumen a la tabla de contenido

Los estudiantes universitarios que hacen un uso excesivo y/o inadecuado del smartphone tienen un menor rendimiento académico e insatisfacción al no gestionar adecuadamente su tiempo e incumplir los plazos de sus actividades pendientes. Dado que el ecosistema de los teléfonos móviles y sus aplicaciones incentiva a pasar el mayor tiempo posible usándolos, se propone desarrollar una pantalla de inicio (launcher) para teléfonos Android que apunte a reducir la procrastinación mediante una interfaz mínimamente llamativa, a mitigar las distracciones y a gestionar mejor el tiempo invertido dentro de las aplicaciones, principalmente en horarios donde se requiera total disposición a una actividad concreta.

\textbf{Palabras clave:} Procrastinación, Uso Excesivo del Smartphone, Gestión del Tiempo, Launcher Android, Productividad.


%%%%%%%%%%%%%%%%%%%%%%%%%%%%% Introducción %%%%%%%%%%%%%%%%%%%%%%%%%%%%%%%%%
\section*{Introducción.}
\addcontentsline{toc}{section}{Introducción.}
En un mundo conectado digitalmente, donde la tecnología tiene cada vez más alcance y uso en todas las actividades de la vida diaria, es muy común emplear un smartphone para realizarlas o complementarlas. Su amplio abanico de herramientas lo hace apropiado para muchos tipos de tareas y esto hace que, en ocasiones, se emplee gran cantidad de tiempo en ellos, generando distracciones casi inevitables. Esto es perjudicial para las personas que necesitan concentrarse lo mejor posible en su jornada laboral o académica y está afectando, en gran medida, a los estudiantes universitarios.

Los dispositivos móviles y las aplicaciones hoy en día (en especial las empresas detrás de ellos) diseñan sus productos con base en ciertos principios y teorías que fomentan un uso adictivo, teniendo como objetivo retener a los usuarios el mayor tiempo posible dentro de las plataformas \cite{Montag2019} y así maximizar sus ganancias dependiendo del modelo de negocio de cada una. Esto es lo que Santiago Giraldo y Cristina Fernández denominan la \textit{economía de la atención} \cite{Giraldo2020}.

Un estudio realizado a 536 estudiantes de educación superior en Estados Unidos revela que aquellos que pasan una cantidad de tiempo prolongada usando el smartphone ven disminuido su rendimiento académico y su nivel de aprendizaje por las constantes distracciones que genera, además de afectar las variables como la motivación y la autorregulación, aspectos clave en la formación de los individuos \cite{Lepp2015}. Además, muchos de los estudiantes no son conscientes de cuánto tiempo pasan en estos dispositivos: su uso se ha vuelto parte de su rutina desde el primer momento del día y se ven en la necesidad de permanecer conectados \cite{Giraldo2020}. Estos factores propician que la situación se vuelva repetitiva y pueda catalogarse como procrastinación, haciendo que los universitarios acumulen y posterguen sus actividades, trayendo consigo dificultades no solo en las aulas de clase, sino también a nivel personal \cite{Kus2016}, físico \cite{Grewal2020,Puerto2015}, emocional y social \cite{Beutel2016}. 

Para minimizar estos comportamientos y analizar a detalle el uso del smartphone, teniendo en cuenta que aproximadamente el 70\% de los usuarios poseedores de un smartphone a nivel mundial tienen instalado el sistema operativo Android \cite{AndroidUsers2024}, se propone desarrollar un launcher para esta plataforma que ayude a los universitarios a gestionar de mejor manera el tiempo que pasan en su smartphone enfocado a disminuir la procrastinación, especialmente en lo académico, y aumentar sus niveles de productividad y bienestar general.



%%%%%%%%%%%%%%%%%%%%%%%%%%%%% Formulación del problema %%%%%%%%%%%%%%%%%%%%%%%%
\section{Formulaci\'on del problema.}

\subsection{Descripci\'on del problema.}

La forma en la que los estudiantes universitarios llevan a cabo su proceso de formación tiene un gran componente autodidacta que requiere de concentración, autorregulación, autoorganización y autoevaluación tanto en el aula de clase como en otros espacios y tiempos destinados al desarrollo de sus actividades académicas \cite{Rafaila2015}. Su proceso fluye normalmente cuando están presentes varios de estos factores, pero la productividad disminuye al tener distractores constantes en el entorno como pueden ser las notificaciones provenientes del smartphone, un ambiente ruidoso en casa o en espacios fuera del aula de clase. De hecho, el aula de clase también puede convertirse en un espacio propicio a distracciones si la temática de la clase es confusa y/o monótona, conduciendo al uso ineludible del smartphone para desconectar del momento.

Estos dispositivos se convierten en un escape, una salida rápida de los momentos difíciles, la soledad, las responsabilidades, entre otros. Además, se expande hasta abarcar una gran porción de tiempo en la vida diaria, reduciendo o eliminando los momentos que se deberían usar para el crecimiento personal y profesional, subestimando los plazos de entrega de sus pendientes pensando que aún hay tiempo suficiente y que se cuenta con la capacidad de elaborarlos, sin pensar en consecuencias futuras \cite{Lizbeth2023}.

A pesar de que muchos de los estudiantes son conscientes del uso excesivo que hacen de las plataformas móviles \cite{Giraldo2020, Beutel2016}, se crea una bola de nieve de insatisfacción, frustración, estrés y ansiedad por postergar sus actividades, no cumplir sus objetivos, no entregar a tiempo sus trabajos y no interiorizar adecuadamente el conocimiento adquirido, haciendo que el rendimiento en su proceso formativo disminuya y su estabilidad emocional se vea afectada. Todos estos sentimientos solo acentúan más el problema de la procrastinación por el uso del smartphone porque se sigue recurriendo a él buscando dopamina que genera, dando paso a una mala gestión del tiempo y deteriorando la calidad de los resultados esperados en sus actividades.

El uso prolongado de plataformas móviles como redes sociales y servicios de streaming afecta negativamente varios aspectos de la vida de los estudiantes incluyendo el rendimiento académico, la salud mental y el bienestar, el desarrollo social y la adicción a los dispositivos. Las empresas que dirigen este sector utilizan estrategias de diseño psicológico, como el uso del color, el texto y la tipografía, para mantener a los usuarios más tiempo en sus aplicaciones, creando un sentido de urgencia y pertenencia a través de notificaciones constantes y actualizaciones frecuentes de contenido y características nuevas, aprovechándose de aspectos primitivos de la misma psicología humana \cite{Neyman2017}.

Preocupa aún más si se tiene en cuenta que, según estadísticas del Ministerio de Educación de Perú, la mayoría de los estudiantes de educación superior, concretamente el 65\%, se encuentra en el rango entre los 18 y 25 años \cite{UniversidadCifras2023}, mismo rango en el cual se determinó que los universitarios tienen un alto grado de procrastinación y que es significativamente mayor que aquellos por encima de los 25 años \cite{Rodriguez2017}.

Por todos los problemas mencionados surge la necesidad de, a través del mismo dispositivo en el que los estudiantes universitarios reinciden en un comportamiento adictivo, regular el uso de las aplicaciones y ayudar a recuperar la excesiva cantidad de tiempo invertido en el smartphone que le corresponde a actividades de mayor importancia, sobre todo a nivel académico y personal.


\subsection{Definici\'on del problema.}

¿Cómo puede una aplicación móvil diseñada para regular el tiempo de uso del smartphone y optimizar la gestión de las asignaciones ayudar a los estudiantes universitarios a minimizar la procrastinación vinculada al uso excesivo de estas plataformas y a mejorar su concentración, autorregulación y productividad en el ámbito académico y personal?

%%%%%%%%%%%%%%%%%%%%%%%%%%%%%% Marco Referencial %%%%%%%%%%%%%%%%%%%%%%%%%%%%%
\section{Marco referencial.}

\subsection{Marco te\'orico.}

\textbf{Procrastinación.} Es el hábito de posponer tareas o responsabilidades cruciales en favor de actividades más inmediatas o gratificantes, aunque menos relevantes a largo plazo. Esta tendencia puede estar influenciada por una variedad de factores psicológicos y ambientales, como la ansiedad ante el fracaso, la falta de motivación intrínseca para completar las tareas asignadas y la presencia constante de distracciones, entre las cuales destaca el uso excesivo del smartphone. La combinación de estos elementos puede llevar a un ciclo perjudicial de aplazamiento continuo, afectando negativamente tanto el rendimiento académico como el bienestar emocional de los estudiantes \cite{Beutel2016}.

\textbf{Tecnología y adicción.} La creciente adicción a la tecnología, particularmente a los dispositivos móviles y sus aplicaciones, ha captado la atención de investigadores y profesionales en salud mental. Estos dispositivos están diseñados con características específicas destinadas a maximizar la participación del usuario, desde las notificaciones constantes hasta la gamificación de las experiencias de usuario. Estas estrategias pueden desencadenar comportamientos adictivos al generar una gratificación instantánea y dificultar la capacidad de autorregulación del tiempo de uso. Como resultado, los usuarios pueden encontrarse atrapados en un ciclo de consumo compulsivo de contenido digital, lo que puede tener repercusiones negativas en su bienestar psicológico y en su capacidad para concentrarse en actividades importantes, como el estudio académico \cite{Montag2019}.

\textbf{Teoría del diseño centrado en el usuario.} Es una metodología orientada a crear productos y servicios que se adapten de manera óptima a las necesidades, preferencias y habilidades de los usuarios finales. El diseño centrado en el usuario implica iteraciones continuas y pruebas de usabilidad para garantizar que el producto final sea fácil de usar y cumpla con las expectativas de los usuarios. Esto implica la creación de prototipos y la realización de pruebas con estudiantes para evaluar la funcionalidad, la accesibilidad y la eficacia del launcher en la mejora de la gestión del tiempo y la reducción de la procrastinación.

\textbf{Psicología del color.} La Psicología del Color resalta cómo los colores influyen en las emociones y el comportamiento humano. En el diseño del launcher, la elección cuidadosa de colores es crucial, ya que puede impactar la motivación, enfoque y compromiso del usuario con las tareas académicas. Los tonos cálidos como el rojo y el naranja pueden estimular la energía y la acción, fomentando la productividad. Mientras tanto, colores suaves como el azul y el verde pueden crear un ambiente tranquilo, ideal para momentos de concentración y estudio. Mantener consistencia en la paleta de colores y su integración con la interfaz del launcher puede mejorar la experiencia del usuario, aumentando su disposición a utilizar la aplicación de manera efectiva para gestionar su tiempo y combatir la procrastinación.

\textbf{Gestión del tiempo.} Es el proceso de planificar, organizar y controlar cómo se utiliza el tiempo disponible para lograr objetivos específicos, tanto personales como profesionales. Implica identificar las tareas prioritarias, asignarles el tiempo adecuado y utilizar técnicas y herramientas para maximizar la eficiencia y la productividad. El launcher podría ofrecer funciones como recordatorios personalizados para tareas importantes, integración de calendario para una visualización clara de horarios y compromisos, y la capacidad de establecer objetivos académicos con seguimiento de progreso. Estas características permitirían a los estudiantes planificar y organizar sus actividades de manera efectiva, evitando conflictos y garantizando el cumplimiento de plazos, mientras identifican áreas de mejora para optimizar su rendimiento académico \cite{Mengual2012}. 

\textbf{Productividad.} Es la capacidad de producir más resultados con la misma cantidad de recursos, o producir los mismos resultados con menos recursos, en un período de tiempo determinado. Esto es esencial para el éxito tanto académico como personal, implicando la eficiencia en la realización de tareas y la obtención de resultados satisfactorios. En el contexto del launcher, se busca potenciar la productividad de los estudiantes mediante la minimización de distracciones y el logro de objetivos de manera efectiva. Esto se logra a través de funciones que permiten enfocarse en tareas prioritarias y limitar las interrupciones, organizar aplicaciones de manera intuitiva para acceder rápidamente a recursos de estudio, y proporcionar herramientas de seguimiento del tiempo para analizar y mejorar la eficiencia en el uso del dispositivo. 


\subsection{Estado del arte.}

El estudio "Adicción a las Redes Sociales y Procrastinación Académica en estudiantes Universitarios" \cite{Luisa2019} encontró una correlación positiva y significativa entre la adicción a las redes sociales y la procrastinación académica, lo que significa que niveles más altos de adicción a las redes sociales corresponden a niveles más altos de procrastinación académica. El estudio también destaca la prevalencia de ambos problemas entre los estudiantes universitarios de todo el mundo, ya que sólo el 15\% de los estudiantes no muestra ningún nivel de adicción a las redes sociales. Los resultados se basan en una muestra de estudiantes de Lima y son relevantes para comprender el impacto de las redes sociales en el rendimiento académico. Este estudio proporciona evidencia de la relación significativa entre la adicción a las redes sociales y la procrastinación académica en estudiantes universitarios, lo cual es una preocupación creciente en la educación superior. Los hallazgos sugieren que abordar la adicción a las redes sociales puede ser una estrategia eficaz para reducir la procrastinación académica y mejorar los resultados de los estudiantes.

El artículo "La Generación Zombie. El uso excesivo de teléfonos celulares en las aulas universitarias peruanas" \cite{Montenegro2023}. Analiza el uso excesivo de celulares por parte de estudiantes universitarios en las aulas de clase y sus efectos negativos en el aprendizaje y la salud. Se destaca que dos tercios de los estudiantes encuestados reconocen que el uso del celular en el aula afecta negativamente su aprendizaje al distraerlos y alejarlos de prestar atención a las actividades académicas. De igual manera resalta que casi el 70\% de los estudiantes está consciente de que el uso excesivo del celular es nocivo para la salud física y mental, generando adicción, problemas de concentración, afectando la interacción humana, etc. Por último, sugieren que las universidades deberían implementar políticas y estrategias para regular el uso de teléfonos móviles en las aulas, como horarios designados para el uso del teléfono y promover el uso responsable.

Adiba Orzikulova destaca los impactos negativos del uso excesivo de teléfonos inteligentes, particularmente en aplicaciones de redes sociales, en los estudiantes universitarios \cite{Orzikulova2022}. El estudio sugiere que las intervenciones de autoayuda basadas en aplicaciones móviles pueden ser efectivas para prevenir el uso excesivo de teléfonos inteligentes y la adicción a los teléfonos inteligentes, pero se necesita más investigación para comprender los mecanismos subyacentes de la adicción a los teléfonos inteligentes y el impacto de estas intervenciones en el comportamiento de los estudiantes. El estudio también enfatiza la importancia de la autorregulación y sugiere que las barreras físicas pueden ser más efectivas para prevenir la adicción a los teléfonos inteligentes y promover la autorregulación que las intervenciones basadas únicamente en software.


\subsection{Marco conceptual.}

\textbf{Aprendizaje autodirigido.} Es un proceso de aprendizaje en el que el estudiante lleva las riendas de su propio proceso educativo. Implica que el alumno identifica de forma autónoma sus necesidades y objetivos de aprendizaje, busca y selecciona los recursos y estrategias más adecuados, y evalúa por sí mismo sus logros y resultados. Requiere que el estudiante emplee habilidades de autorregulación, como estrategias cognitivas, metacognitivas y de motivación personal, ya sea trabajando de manera independiente o con cierta guía externa. En esencia, el aprendiz toma un papel activo y autorresponsable en la conducción de su aprendizaje \cite{Marquez2014}.

\textbf{Launcher Android.} Un launcher en Android es una aplicación que permite personalizar la interfaz de usuario y la apariencia de la pantalla de inicio de un dispositivo móvil. Funciona como una capa de personalización sobre el sistema operativo Android, permitiendo al usuario modificar la disposición de iconos, widgets, fondos de pantalla y otros elementos visuales en la pantalla de inicio. Los launchers ofrecen opciones de personalización avanzadas, como cambiar el diseño de la pantalla de inicio, agregar efectos de transición, modificar los iconos de las aplicaciones y ajustar la organización de las aplicaciones \cite{Launcher}.

\textbf{Autocontrol.} Capacidad de modular y controlar las propias acciones de una forma apropiada a la edad de la persona. Es considerado uno de los componentes clave de la inteligencia emocional que debe ser reeducado en los estudiantes. El autocontrol implica tener una sensación de control interno sobre el propio cuerpo, conducta y entorno, permitiendo regular los impulsos y actuar de manera intencionada para lograr los objetivos deseados. Se plantea que el desarrollo del autocontrol desde la infancia constituye una facultad fundamental en el ser humano para tener una voluntad sólida y capacidad de autogobernarse \cite{Navarro2003}. 

\textbf{Rendimiento académico.} El rendimiento académico se refiere a la evaluación y medición de los conocimientos y habilidades adquiridos por un estudiante durante su formación académica, ya sea en niveles escolares, terciarios o universitarios. Se considera que un alumno tiene un buen rendimiento académico cuando obtiene calificaciones positivas y aprobatorias en los exámenes y evaluaciones que rinde a lo largo de los ciclos o períodos académicos correspondientes \cite{RendimientoEscolar}.

\textbf{Concentración.} La concentración es el enfoque voluntario de la mente hacia un objetivo específico, tarea o actividad, excluyendo cualquier distracción o interferencia que pueda surgir. Es un proceso psíquico que se lleva a cabo mediante el razonamiento, donde se dirige toda la atención hacia un punto determinado, ya sea una tarea en curso o algo en lo que se está pensando \cite{Concentracion2023}.

\textbf{Frustración.} La frustración es un estado emocional caracterizado por sentimientos de decepción, irritabilidad o insatisfacción que surgen cuando una persona experimenta obstáculos o dificultades para alcanzar sus metas o satisfacer sus necesidades. Se manifiesta como una respuesta emocional negativa frente a la percepción de que los esfuerzos realizados no han dado los resultados deseados. La frustración puede surgir en diversas situaciones de la vida cotidiana, como problemas laborales, académicos, personales o sociales, y puede tener efectos adversos en el bienestar emocional y el comportamiento de la persona afectada \cite{Kamenetzky2009}.

%%%%%%%%%%%%%%%%%%%%%%%%%%%%%% Alcance %%%%%%%%%%%%%%%%%%%%%%%%%%%%%%
\section{Alcance.}	

\subsection{Declaración del alcance.}

El alcance del presente trabajo de grado consiste en el desarrollo de un launcher Android personalizado para estudiantes de la Universidad del Valle en Tuluá que incorpore las siguientes funcionalidades: 

\begin{enumerate}
    \item \textbf{Diseño minimalista:} Contará con una interfaz sin íconos de aplicaciones para evitar desviar la atención, en tonos neutrales y con atajos útiles para gestionar las demás características presentes en el launcher.
    \item \textbf{Seguimiento del tiempo de uso de las aplicaciones:} El tiempo que se pasa en ciertas aplicaciones seleccionadas por el estudiante se podrá regular y monitorear con el fin de registrar progreso o áreas de mejora.
    \item \textbf{Gestión de tareas y hábitos:} El launcher permitirá consignar las tareas que se deseen realizar y crear hábitos en función de tareas, tiempo límite o chequeo regular del hábito en cuestión.
    \item \textbf{Herramienta(s) de productividad:} Se integrará al menos una herramienta de productividad para apoyar la realización de las tareas o los hábitos, como técnicas de distribución del tiempo de sesiones de trabajo, priorización de tareas, entre otras.
\end{enumerate}

%%%%%%%%%%%%%%%%%%%%%%%%%%%%%%%%%%%%%%%%%%%%

\subsection{Supuestos.}

\begin{enumerate}
    \item Correcto funcionamiento de los equipos donde se llevará a cabo el desarrollo y las pruebas del proyecto. 
    \item Normal desarrollo de los periodos académicos.
    \item Continuidad laboral del director de trabajo de grado.
\end{enumerate}

%%%%%%%%%%%%%%%%%%%%%%%%%%%%%%%%%%%%%%%%%%%%

\subsection{Restricciones.}

\begin{enumerate}
    \item El proyecto se debe llevar a cabo en un plazo máximo de ocho (8) meses.
    \item Solo estará disponible para el sistema operativo Android.
    \item Será diseñado para el tamaño y la interfaz de un smartphone, no para tablets u otros dispositivos similares.
    \item El proyecto y anteproyecto deben ser aprobados.
\end{enumerate}



%%%%%%%%%%%%%%%%%%%%%%%%%%%%%% Objetivos %%%%%%%%%%%%%%%%%%%%%%%%%
\section{Objetivos.}

\subsection{Objetivo general.}

Desarrollar un launcher Android personalizado para regular el uso del Smartphone y aumentar la productividad de los estudiantes de la Universidad del Valle en Tuluá.

\subsection{Objetivos espec\'ificos.}	

\begin{enumerate}
    \item Definir los requisitos funcionales de una pantalla de inicio para Android.
    \item Diseñar la arquitectura interna e interfaces gráficas de la aplicación.
    \item Integrar el diseño y las funcionalidades en la aplicación de acuerdo con los requisitos establecidos.
    \item Implementar pruebas unitarias y de usabilidad del launcher en un sistema Android.
\end{enumerate}

\subsection{Resultados esperados.}	

\begin{table}[H]
\caption{Resultados esperados}
\begin{tabular*}{\textwidth}{|p{0.5\textwidth}|p{0.5\textwidth}|}
\cline{1-2}
\multicolumn{1}{|c|}{\cellcolor[gray]{0.9} \textbf{Objetivo específico}} &  \multicolumn{1}{|c|}{\cellcolor[gray]{0.9} \textbf{Resultados esperados}}  \\
\cline{1-2}
1. Definir los requisitos funcionales de una pantalla de inicio para Android. & 
Documento donde se describan las secciones y herramientas que incorporará, lenguajes y tecnologías a emplear.
\\
\cline{1-2} 

2. Diseñar la arquitectura interna e interfaces gráficas de la aplicación. & 
Diseño de las vistas de la aplicación con sus elementos e interacciones como un prototipo; especificación de la arquitectura que se usará para la estructuración del código y los componentes del launcher.
\\

\cline{1-2} 

3. Integrar el diseño y las funcionalidades en la aplicación de acuerdo con los requisitos establecidos. & 
Código fuente del launcher con el diseño y las características definidas con anterioridad. \\

\cline{1-2} 

4. Implementar pruebas unitarias y de usabilidad del launcher en un sistema Android. & 
Informe de las pruebas a nivel de código y de experiencia de usuario del launcher instalado en un dispositivo Android. \\

\cline{1-2} 

\end{tabular*}
\end{table}







%%%%%%%%%%%%%%%%%%%%%%%%%%%%%% Metodología %%%%%%%%%%%%%%%%%%%%%%%%%
% \pagebreak
% \section{Metodología.}

\section{Metodología.}

\subsection{Tecnologías empleadas.}

En el panorama actual del desarrollo móvil, los desarrolladores se enfrentan a una decisión fundamental entre el desarrollo nativo y multiplataforma. Con el crecimiento exponencial del mercado de aplicaciones móviles y la demanda de experiencias de usuario cada vez más sofisticadas, la elección tecnológica se convierte en un factor determinante para el adecuado desarrollo del proyecto. Así pues, se realizó una evaluación de las tecnologías disponibles, considerando las necesidades específicas y las tendencias actuales de la industria.


\subsubsection{Desarrollo nativo vs. multiplataforma.}

Se optó por el desarrollo nativo de la aplicación ya que ofrece mejor rendimiento, fácil acceso a todos los recursos del smartphone y está enfocado a un sistema operativo en particular (Android). Los frameworks multiplataforma actuales también ofrecen buen rendimiento e integración con las herramientas de desarrollo de Android, pero son más propensos a tener menor rendimiento, usar más recursos del sistema y generar fallos debido a las diferencias entre iOS y Android.

Esta decisión se fundamenta en la naturaleza específica del launcher, que requiere integración profunda con el sistema operativo Android para gestionar aplicaciones, controlar tiempos de uso y proporcionar una experiencia de usuario fluida y responsiva. Los resultados la comparativa se consignaron en la Tabla \ref{tab:comparacion_nativo_multiplataforma}.

\begin{table}[H]
\centering
\caption{Comparación entre desarrollo nativo y multiplataforma para Android}
\label{tab:comparacion_nativo_multiplataforma}
\begin{tabular}{|p{0.25\textwidth}|p{0.35\textwidth}|p{0.35\textwidth}|}
\hline
\textbf{Aspecto}{\cellcolor[gray]{0.9}} & \textbf{Desarrollo Nativo Android}{\cellcolor[gray]{0.9}} & \textbf{Desarrollo Multiplataforma}{\cellcolor[gray]{0.9}} \\
\hline
Lenguaje de programación & Kotlin o Java & JavaScript (React Native), Dart (Flutter), C\# (Xamarin) \\
\hline
Rendimiento & Óptimo, optimizado específicamente para Android & Bueno, pero generalmente inferior debido a la capa de abstracción \\
\hline
Acceso a funcionalidades & Acceso total a todas las APIs y hardware específicos & Acceso limitado, requiere plugins para funcionalidades específicas \\
\hline
Experiencia de usuario & Adaptación completa a Material Design & Puede no seguir completamente las directrices de Android \\
\hline
Tiempo de desarrollo & Puede ser más largo debido a codificación específica & Más corto, código compartido entre plataformas \\
\hline
Costos de desarrollo & Más altos para Android únicamente, pero justificados & Más bajos si se desarrolla para múltiples plataformas \\
\hline
Mantenimiento & Simplificado, enfoque exclusivo en Android & Más complejo para compatibilidad específica \\
\hline
Actualizaciones de SO & Implementación inmediata con nuevas versiones & Depende de actualizaciones del framework \\
\hline
Reutilización de código & Nula entre plataformas, alta en proyectos Android & Alta reutilización multiplataforma \\
\hline
Compatibilidad & Total, diseño específico para Android & Buena, pero con posibles inconsistencias \\
\hline
\end{tabular}
\end{table}

\subsubsection{Versión objetivo de la API de Android.}

La elección de Android 10 (API 29) como versión mínima para el desarrollo de Dino Launcher se fundamenta en la convergencia entre las funcionalidades que introduce esta versión del sistema operativo, los requisitos específicos del launcher desarrollado y la posibilidad de que pueda ser ejecutado en dispositivos no tan recientes de manera satisfactoria. En la figura \ref{fig:versiones_android}, los datos de distribución de Android 10 alcanzan el 81.2\% de los dispositivos en el mercado, lo que garantiza una amplia compatibilidad con los dispositivos de los usuarios objetivo. 

\begin{figure}[H]
\caption{Distribución de versiones de Android a nivel mundial. \cite{AndroidStudio}}
\label{fig:versiones_android}
\includegraphics[width=\textwidth]{Figuras/versiones_android.png}
\centering
\end{figure}

Esta versión introduce características esenciales que hacen posible una de las funcionalidades centrales del launcher, la introducción nativa del \textbf{modo oscuro}. Esta permite al launcher mostrar automáticamente los modos claro y oscuro dependiendo de la configuración del tema del sistema, reduciendo significativamente la fatiga visual durante el uso del launcher y mejorando la experiencia de usuario, especialmente durante las horas nocturnas cuando el control del tiempo de pantalla es más crítico. 

\subsubsection{Selección del lenguaje: Kotlin.}

Las dos alternativas principales para el desarrollo nativo en Android son Java y Kotlin. Aunque Java es el lenguaje tradicional para el desarrollo Android, ha perdido terreno gracias a Kotlin y sus mejoras significativas con respecto a Java para el desarrollo móvil, principalmente en cuanto a sintaxis y características modernas. Según la encuesta anual de desarrolladores de Stack Overflow de mayo de 2024 \cite{Stackoverflow2024}, los desarrolladores que trabajan con Kotlin se sienten más cómodos con el lenguaje, al contrario de lo que se observa en Java, donde varios de los encuestados preferirían trabajar con Kotlin. La selección de Kotlin se fundamenta en los siguientes aspectos:

\begin{itemize}
    \item Google, propietario de Android, recomienda Kotlin para cualquier proyecto nuevo de Android y declaró que construiría sus herramientas de desarrollo con un enfoque Kotlin-first desde la conferencia Google I/O en 2019 \cite{Google2019}.
    \item Es un lenguaje más fácil de entender por su sintaxis simplificada y la reducción de código repetitivo (\textit{boilerplate}), reduciendo el tiempo de aprendizaje.
    \item Es un lenguaje activamente desarrollado y adaptado a las necesidades actuales de la industria.
    \item Tiene una gran comunidad y cantidad de recursos disponibles.
    \item Ofrece características modernas como corrutinas, funciones de extensión y clases de datos que facilitan el desarrollo de aplicaciones complejas.
\end{itemize}

La Tabla \ref{tab:comparacion_kotlin_java} resume las principales diferencias entre Kotlin y Java, destacando las ventajas de Kotlin para el desarrollo del launcher.

\begin{table}[ht]
\centering
\caption{Comparación entre Kotlin y Java para desarrollo Android}
\label{tab:comparacion_kotlin_java}
\begin{tabular}{|p{0.25\textwidth}|p{0.35\textwidth}|p{0.35\textwidth}|}
\hline
\textbf{Aspecto}{\cellcolor[gray]{0.9}} & \textbf{Kotlin}{\cellcolor[gray]{0.9}} & \textbf{Java}{\cellcolor[gray]{0.9}} \\
\hline
Año de lanzamiento & 2011 & 1995 \\
\hline
Sintaxis & Concisa, moderna y más legible & Extensa, más detallada y tradicional \\
\hline
Interoperabilidad & Totalmente interoperable con Java & No es nativamente interoperable con Kotlin \\
\hline
Seguridad de tipos nulos & Evita NullPointerExceptions & NullPointerExceptions son comunes \\
\hline
Compatibilidad Android & Totalmente compatible, lenguaje oficial & Compatible pero no recomendado oficialmente \\
\hline
Características modernas & Lambdas, corrutinas, extension functions, data classes & Introducción más lenta de características modernas \\
\hline
Curva de aprendizaje & Relativamente fácil para desarrolladores Java & Relativamente fácil pero más verboso \\
\hline
Productividad & Alta, gracias a sintaxis concisa & Moderada, requiere más código para tareas similares \\
\hline
Soporte y comunidad & Creciente, especialmente en Android & Muy grande y establecida, décadas de documentación \\
\hline
Desempeño & Similar a Java, optimizaciones específicas & Similar a Kotlin, rendimiento comparable \\
\hline
Ecosistema de herramientas & Totalmente soportado en Android Studio & Amplio soporte en diversas IDEs \\
\hline
\end{tabular}
\end{table}

\subsubsection{Librería de UI: Jetpack Compose.}

Jetpack Compose es el toolkit de UI más reciente y recomendado por Google para el desarrollo de interfaces en Android desde 2021, marcando un cambio paradigmático hacia la programación declarativa. Compose permite construir interfaces de usuario de manera más intuitiva, eficiente y mantenible en comparación con el sistema tradicional basado en vistas XML. Su adopción se ha consolidado como el estándar para nuevos proyectos Android, facilitando la creación de experiencias visuales modernas y adaptables.

El aspecto más relevante de Jetpack Compose es su integración nativa con Kotlin, lo que permite aprovechar plenamente las características modernas del lenguaje como las funciones lambda, la programación funcional y la sintaxis concisa, mejorando la legibilidad y mantenibilidad del código y reduciendo la cantidad de código necesario para construir interfaces. En la Figura \ref{fig:compose_vs_xml}, se muestra un ejemplo de código en Jetpack Compose que ilustra su simplicidad y claridad en comparación con el enfoque tradicional de XML para construir una interfaz que muestre las palabras \textit{"Good"} \textit{"Morning"} una arriba de la otra.

\begin{figure}[ht]
\caption{Ejemplo de código en Jetpack Compose}
\label{fig:compose_vs_xml}
\includegraphics[width=\textwidth]{Figuras/compose_vs_xml.jpg}
\centering
\end{figure}

\pagebreak

La Tabla \ref{tab:comparacion_compose_xml} presenta una comparación detallada entre Jetpack Compose y el sistema tradicional de vistas XML, destacando las ventajas técnicas que justifican la adopción de Compose para el desarrollo del launcher.

\begin{table}[H]
\centering
\caption{Comparación entre Jetpack Compose y vistas XML para desarrollo Android}
\label{tab:comparacion_compose_xml}
\begin{tabular}{|p{0.25\textwidth}|p{0.35\textwidth}|p{0.35\textwidth}|}
\hline
\textbf{Aspecto}{\cellcolor[gray]{0.9}} & \textbf{Jetpack Compose}{\cellcolor[gray]{0.9}} & \textbf{Vistas XML}{\cellcolor[gray]{0.9}} \\
\hline
Paradigma de programación & Declarativo, describe qué mostrar & Imperativo, describe cómo construir \\
\hline
Lenguaje & 100\% Kotlin, fuertemente tipado & XML + Kotlin/Java, tipos débiles \\
\hline
Gestión de estado & Recomposición automática con State & Actualización manual de vistas \\
\hline
Animaciones & APIs nativas integradas, declarativas & Requiere configuración compleja XML/código \\
\hline
Reutilización de código & Composición funcional, alta reutilización & Herencia de vistas, reutilización limitada \\
\hline
Curva de aprendizaje & Requiere conocimiento de programación funcional & Familiar para desarrolladores tradicionales \\
\hline
Rendimiento & Recomposición inteligente, optimizado & Inflado de layouts puede ser costoso \\
\hline
Depuración & Compose Inspector integrado & Layout Inspector tradicional \\ 
\hline
Tamaño de APK & Comparable o menor & Puede ser mayor con layouts complejos \\
\hline
Interoperabilidad & Completa con vistas XML existentes & No compatible con Compose sin wrappers \\
\hline
Soporte de temas & Material Design 3 nativo & Requiere configuración manual extensa \\
\hline
Testing & Testing declarativo con ComposeTestRule & Testing complejo de interacciones UI \\
\hline
Mantenimiento & Simplificado, lógica en un solo lugar & Complejo, lógica distribuida en XML y código \\
\hline
Adopción empresarial & Creciente, recomendado por Google & Establecido, pero en desuso gradual \\
\hline
\end{tabular}
\end{table}

\subsubsection{Persistencia de datos: Room.}

Room es una biblioteca de persistencia que forma parte de Android Jetpack \cite{Jetpack} y proporciona una capa de abstracción sobre SQLite. Está diseñada específicamente para simplificar el trabajo con bases de datos en Android, combinando la potencia de SQLite con la comodidad y seguridad de tipos del desarrollo en Kotlin \cite{Room}. Entre las principales ventajas de utilizar Room en el desarrollo Android, se encuentran la posibilidad de observar cambios en los datos de manera reactiva con LiveData y Flow, facilitando la actualización automática de la interfaz de usuario; el soporte nativo para corrutinas, que permiten la ejecución de manera asíncrona utilizando \textit{suspend functions}, evitando bloqueos en el hilo principal y el uso de anotaciones para reducir código repetitivo (como \texttt{@Entity}, \texttt{@Dao} o \texttt{@Database}) para generar automáticamente el código necesario y realizar operaciones en SQLite. En la figura \ref{fig:room}, se muestra el flujo de datos de Room entre la base de datos y la aplicación desarrollada.

\begin{figure}[ht]
\caption{Flujo de datos de Room. \cite{Room}}
\label{fig:room}
\includegraphics[width=\textwidth]{Figuras/room.png}
\centering
\end{figure}

La selección de Room como solución de persistencia de datos por encima de alternativas basadas en la nube, como Firebase, se fundamenta en las características específicas del launcher y sus requerimientos funcionales en la sección \ref{chap:requerimientos_funcionales}. El proyecto no requiere que los datos estén disponibles en múltiples dispositivos o se sincronicen en tiempo real, ya que el launcher está diseñado para funcionar de manera individual en cada dispositivo. Además, el tipo de información que maneja la aplicación no demanda características de red en tiempo real ni almacenamiento de datos complejos o de gran tamaño. La naturaleza personal y privada de estos datos hace innecesaria la implementación de sistemas de autenticación de cuentas de usuario o gestión de perfiles en la nube, simplificando la arquitectura del sistema y minimizando el uso de recursos que puedan llegar a consumir las funciones de red.


\subsection{Levantamiento de requerimientos.}

Para llevar a cabo el desarrollo del launcher Android enfocado en combatir la procrastinación en estudiantes universitarios, se utilizaron múltiples técnicas de recolección de información que representaran de una manera clara las acciones a realizar para cumplir los objetivos, desde la teoría hasta la manera en cómo se interactúa con los dispositivos móviles. A continuación, se detallan las técnicas empleadas y los resultados obtenidos:

\subsubsection{Metodología empleada.}

\textbf{Lluvia de ideas (Brainstorming):} Se recurrió a la técnica de brainstorming para generar un amplio abanico de ideas sobre las funcionalidades que el launcher podría ofrecer. A través de sesiones creativas, se exploraron diversas posibilidades y se identificaron las características más importantes que podrían contribuir a mejorar la productividad de los estudiantes. Esta técnica permitió establecer el conjunto inicial de funcionalidades como la gestión de tareas, gestión de hábitos y control de tiempo de uso de aplicaciones, en conjunto con sus particularidades a nivel de desarrollo y diseño.

\textbf{Observación:} Al analizar la interacción de los estudiantes con sus dispositivos móviles en entornos académicos, especialmente de la Universidad del Valle en Tuluá, se identificó que muchos de ellos utilizan aplicaciones de mensajería, redes sociales y juegos como una forma de distracción y abren dichas aplicaciones muchas veces por mera memoria muscular. Esta observación llevó a la conclusión de que un launcher que si, a nivel visual, reduce los estímulos que faciliten la detección de dichas aplicaciones, puede contribuir a que sean usadas con menor frecuencia. Es por esto que se definió que la interfaz debía ser minimalista, dejando de lado los íconos que caracterizan a cada aplicación por considerarse un disparador que apunta hacia el inmediato uso de la misma. Este proceso también reafirmó la necesidad de implementar un limite de tiempo a aplicaciones que el usuario identifique que necesita regular.

\textbf{Análisis de documentación:} Se revisaron cuatro artículos relacionados con la procrastinación académica, los cuales sirvieron de apoyo para entender mejor el contexto en el que se desarrollaría el launcher.

Durante el análisis se identificaron patrones de comportamiento que van más allá del simple uso de dispositivos móviles. Los estudios revisados revelaron que existe una relación negativa entre la procrastinación académica y las intenciones de llevar a cabo conductas saludables, asociada con una menor autoeficacia específica de la salud y bajo control conductual percibido \cite{Vanguardia2012}. Se encontró evidencia de una correlación directa de intensidad moderada a fuerte entre la postergación de actividades académicas y la dependencia al dispositivo móvil, siendo especialmente notable en estudiantes universitarios donde el 59,3\% presenta niveles moderados de dependencia \cite{Villagomez2023}.

La investigación también demostró que el uso problemático del smartphone va más allá de las aplicaciones de mensajería y redes sociales, extendiéndose a lo que se denomina "procrastinación electrónica", que abarca el uso excesivo de computadoras, videojuegos, televisión, películas y consumo de noticias. Esta forma de procrastinación resulta particularmente seductora debido a la accesibilidad constante que proporciona internet, permitiendo la distracción en cualquier momento del día.

Los hallazgos confirman que a mayor uso problemático del smartphone, mayor es la tendencia a procrastinar en los estudiantes, y dado que este uso parece estar relacionado con un bajo autocontrol, se identificó la necesidad de implementar programas de intervención relacionados con la resiliencia para controlar el uso del smartphone y mejorar la gestión del tiempo. Estos insights fueron fundamentales para definir las funcionalidades del launcher, especialmente las relacionadas con el control de tiempo de uso y el modo concentración.

La combinación de estas técnicas permitió construir una base sólida para el desarrollo del launcher. Los resultados obtenidos a través de este proceso de levantamiento de requerimientos fueron fundamentales para definir los requisitos funcionales y no funcionales del sistema, asegurando así que el producto final respondiera de manera efectiva a las necesidades de los usuarios.

\subsubsection{Requerimientos funcionales.}

Los requerimientos funcionales identificados se organizaron en las siguientes categorías principales:

\begin{enumerate}
    \item \textbf{Lanzador de aplicaciones:} Funcionalidad principal del launcher que mostrará la lista de aplicaciones disponibles para su ejecución. Incluirá gestión básica de aplicaciones mediante pulsación prolongada para desinstalar, acceder a información de la aplicación o fijar aplicaciones en el escritorio.
    
    \item \textbf{Accesos rápidos:} Sistema de accesos directos para aplicaciones esenciales predeterminadas como correo electrónico,cámara o teléfono, sin necesidad de búsqueda. Permitirá la personalización mediante el fijado de aplicaciones al escritorio.
    
    \item \textbf{Búsqueda integrada:} Barra de búsqueda que mostrará aplicaciones que coincidan con el texto ingresado, así como configuraciones específicas del launcher.
    
    \item \textbf{Modo concentración:} Sistema de bloqueo temporal de aplicaciones que impedirá su uso hasta la desactivación manual o automática del modo. Incluirá bloqueo de notificaciones de aplicaciones restringidas y programación automática basada en horarios académicos.
    
    \item \textbf{Control de tiempo de uso:} Funcionalidad para establecer límites de tiempo en aplicaciones específicas. Una vez agotado el tiempo permitido, se implementará un período de espera antes de permitir el uso nuevamente.
    
    \item \textbf{Gestión de tareas:} Sistema tipo to-do que permitirá asignar fechas específicas a las tareas. Las tareas sin fecha asignada aparecerán diariamente hasta su completación. Incluirá sistema de etiquetas para clasificación y organización.
    
    \item \textbf{Gestión de hábitos:} Implementación de tareas recurrentes programadas para días específicos de la semana, con fechas de inicio y fin definidas. Permitirá marcado de completación similar al sistema de tareas.
    
    \item \textbf{Integración Pomodoro:} Herramienta de productividad configurable que permitirá establecer número de sesiones, duración de sesiones de trabajo y tiempos de descanso.
\end{enumerate}

\subsubsection{Requerimientos no funcionales.}

Los requerimientos no funcionales establecidos para el proyecto incluyen:

\begin{itemize}
    \item \textbf{Compatibilidad:} El launcher debe ser compatible exclusivamente con el sistema operativo Android, diseñado específicamente para smartphones (no tablets).
    
    \item \textbf{Rendimiento:} Debe ofrecer un rendimiento óptimo aprovechando las ventajas del desarrollo nativo en Android.
    
    \item \textbf{Usabilidad:} Interfaz minimalista con diseño centrado en el usuario, siguiendo las directrices de Material Design de Android.
    
    \item \textbf{Seguridad:} Implementación de controles de acceso para las funcionalidades de bloqueo y restricción de aplicaciones.
    
    \item \textbf{Mantenibilidad:} Código estructurado y documentado para facilitar futuras actualizaciones y mejoras.
\end{itemize}

\subsubsection{Investigación tecnológica.}

Como parte del proceso de levantamiento de requerimientos, se realizó una investigación exhaustiva para determinar las tecnologías más apropiadas para el desarrollo del launcher.

\textbf{Decisión de desarrollo nativo vs. multiplataforma:}

Se optó por el desarrollo nativo de la aplicación ya que ofrece mejor rendimiento, fácil acceso a todos los recursos del smartphone y está enfocado a un sistema operativo en particular (Android). Los frameworks multiplataforma actuales también ofrecen buen rendimiento e integración con las herramientas de desarrollo de Android, pero son más propensos a tener menor rendimiento, usar más recursos del sistema y generar fallos debido a las diferencias entre iOS y Android.

\begin{table}[H]
\centering
\caption{Comparación desarrollo nativo vs. multiplataforma}
\begin{tabular}{|p{0.25\textwidth}|p{0.35\textwidth}|p{0.35\textwidth}|}
\hline
\textbf{Aspecto} & \textbf{Desarrollo Nativo Android} & \textbf{Desarrollo Multiplataforma} \\
\hline
Lenguaje & Kotlin o Java & JavaScript, Dart, C\# \\
\hline
Rendimiento & Óptimo, optimizado para Android & Bueno, pero inferior debido a abstracción \\
\hline
Acceso a APIs & Total acceso a APIs y hardware & Limitado, requiere plugins específicos \\
\hline
Experiencia UX & Adaptación completa a Material Design & Puede no seguir completamente las directrices \\
\hline
Mantenimiento & Simplificado para Android & Más complejo para compatibilidad específica \\
\hline
\end{tabular}
\end{table}

\textbf{Selección del lenguaje de programación:}

Las dos alternativas de desarrollo nativo en Android son Java y Kotlin. Java es el lenguaje tradicional para el desarrollo Android, pero ha perdido terreno gracias a Kotlin y sus mejoras con respecto a Java para el desarrollo móvil, principalmente en cuanto a sintaxis. Se seleccionó Kotlin por las siguientes razones:

\begin{itemize}
    \item Google recomienda Kotlin para cualquier proyecto nuevo de Android y declaró un enfoque Kotlin-first desde 2019.
    \item Sintaxis más concisa y moderna que reduce el código boilerplate.
    \item Seguridad de tipos nulos que evita NullPointerExceptions.
    \item Totalmente interoperable con Java.
    \item Comunidad activa y gran cantidad de recursos disponibles.
\end{itemize}

\begin{table}[H]
\centering
\caption{Comparación Kotlin vs. Java}
\begin{tabular}{|p{0.25\textwidth}|p{0.35\textwidth}|p{0.35\textwidth}|}
\hline
\textbf{Aspecto} & \textbf{Kotlin} & \textbf{Java} \\
\hline
Sintaxis & Concisa, moderna y legible & Verbosa y tradicional \\
\hline
Seguridad de tipos nulos & Evita NullPointerExceptions & NullPointerExceptions comunes \\
\hline
Compatibilidad Android & Lenguaje oficial recomendado & Compatible pero no recomendado \\
\hline
Características modernas & Coroutines, extension functions, data classes & Introducción más lenta de características \\
\hline
Productividad & Alta debido a sintaxis concisa & Moderada, requiere más código \\
\hline
\end{tabular}
\end{table}

\textbf{Tecnologías complementarias:}

Para el almacenamiento de datos se seleccionó SQLite como base de datos local, considerando su integración nativa con Android y su eficiencia para el manejo de datos de tareas, hábitos y configuraciones del usuario.

\textbf{Versión mínima de Android:}

Considerando que aproximadamente el 70\% de los usuarios poseedores de un smartphone a nivel mundial tienen instalado el sistema operativo Android, se estableció como requisito técnico soportar las versiones de Android más utilizadas para maximizar la compatibilidad y alcance del launcher.

%%%%%%%%%%%%%%%%%%%%%%%%%%%%%% Arquitectura y diseño %%%%%%%%%%%%%%%%%%%%%%%%%


\section{Arquitectura y diseño.}

\subsection{Arquitectura.}

\subsubsection{Prácticas recomendadas por Google.}

El desarrollo de aplicaciones Android modernas requiere de una arquitectura clara, modular y escalable, que facilite el mantenimiento del código, la incorporación de nuevas funcionalidades, así como la mejora continua del producto final. Google establece principios fundamentales para el desarrollo de aplicaciones Android robustas y mantenibles a través de la plataforma Android Jetpack \cite{Jetpack}, que han evolucionado a partir de años de experiencia en el ecosistema móvil. Estas prácticas se centran en la separación clara de responsabilidades a través de una \textit{arquitectura basada en capas}, donde cada componente del sistema tiene un propósito específico y bien definido \cite{AndroidBestPractices}. 

La arquitectura recomendada para aplicaciones Android se basa principalmente en tres capas fundamentales:


\begin{itemize}
  \item \textbf{Capa de Presentación (UI):} Encargada exclusivamente de la interacción visual y la experiencia del usuario. Esta capa se ocupa de mostrar la información, escuchar las acciones del usuario y reaccionar a los cambios en el estado de la aplicación. No contiene lógica de negocio ni interacción directa con fuentes de datos, garantizando así una clara separación de responsabilidades.

  \item \textbf{Capa de Dominio (opcional, pero recomendada en proyectos de mediana o gran escala):} Constituye el núcleo lógico del sistema, implementando casos de uso específicos relacionados directamente con las reglas del negocio. Al aislar esta lógica en una capa independiente, se favorece la reutilización de código, la claridad en la implementación y se simplifican significativamente las pruebas unitarias.

  \item \textbf{Capa de Datos:} Responsable de acceder a las fuentes de información, tanto locales como remotas. Utiliza patrones de diseño como el repositorio, el cual se encarga de gestionar una fuente única de verdad para los datos, manteniendo copias locales y actualizando la información desde fuentes externas según sea necesario.
\end{itemize}

Dentro de estas prácticas es especialmente importante destacar el patrón conocido como Flujo Unidireccional de Datos \textit{(Unidirectional Data Flow, UDF)}. En el contexto específico de Android, los datos generalmente fluyen desde elementos con un alcance amplio, como los repositorios y fuentes de datos, hacia elementos con un alcance más limitado, como los componentes visuales de la UI. Por otro lado, los eventos generados por el usuario, como pulsaciones de botones o gestos, fluyen en sentido contrario desde la interfaz gráfica hacia la fuente única de verdad, donde los datos se modifican y actualizan posteriormente, siempre a través de estructuras de datos inmutables.

Este enfoque está estrechamente relacionado con la correcta gestión del ciclo de vida de los componentes Android. El conocimiento explícito y el manejo consciente del ciclo de vida permiten que los componentes reaccionen apropiadamente frente a eventos generados por el propio sistema operativo, como la rotación de pantalla o la suspensión temporal de la aplicación. Herramientas proporcionadas por Android Jetpack, como \textit{ViewModel}, y bibliotecas reactivas como \textit{LiveData} o \textit{Flow}, son clave en este proceso, pues reaccionan de forma automática y segura a los cambios de estado en las actividades y fragmentos.

Google también recomienda emplear bibliotecas especializadas como \textit{Hilt}, para gestionar las dependencias entre componentes. La inyección de dependencias automatiza la creación de objetos en tiempo de compilación sin necesidad de que cada clase los construya. En su lugar, una clase externa se encarga de proveer las dependencias requeridas, lo que simplifica la gestión de dependencias, centraliza las instancias de los objetos y mejora la mantenibilidad del código.


\subsubsection{Patrón de arquitectura MVVM.}

El patrón de arquitectura elegido para el desarrollo del launcher fue \textit{Model-View-ViewModel} (\textit{MVVM}, por sus siglas en inglés), ya que su principal objetivo es separar la lógica de negocio de la interfaz de usuario, obedeciendo a las prácticas recomendadas por Google para el desarrollo de aplicaciones Android \cite{AndroidBestPractices}. Aunque originalmente fue diseñado por Microsoft en 2005 para aplicaciones de escritorio basadas en eventos, con el tiempo ha sido cada vez más implementado en el desarrollo de aplicaciones móviles, especialmente con la introducción de \textit{Android Architecture Components} en la conferencia de Google I/O en 2017 \cite{ArchitectureComponents}. Está inspirado en el patrón \textit{Model-View-Presenter} (\textit{MVP}) y \textit{Model-View-Controller} (\textit{MVC}), manteniendo la separación de responsabilidades e introduciendo un enfoque más reactivo y menos acoplado entre la vista y el modelo. Las principales diferencias entre estos patrones arquitectónicos se presentan en la Tabla \ref{tab:comparacion_arquitecturas}.

\begin{table}[ht]
\centering
\caption{Comparación de patrones arquitectónicos: MVC, MVP y MVVM}
\label{tab:comparacion_arquitecturas}
\begin{tabular}{|p{0.2\textwidth}|p{0.25\textwidth}|p{0.25\textwidth}|p{0.25\textwidth}|}
\hline
\textbf{Aspecto} & \textbf{MVC (Model-View-Controller)} & \textbf{MVP (Model-View-Presenter)} & \textbf{MVVM (Model-View-ViewModel)} \\
\hline
\textbf{Acoplamiento} & View y Model están acoplados. Controller gestiona la interacción. & View y Model están completamente desacoplados por el Presenter. & View y ViewModel están débilmente acoplados a través de enlaces de datos (data binding). \\
\hline
\textbf{Rol de View} & Activa. Puede observar directamente a Model y recibir actualizaciones de Controller. & Pasiva. Implementa una interfaz y no contiene lógica. Presenter indica qué mostrar. & Reactiva. Se enlaza a propiedades de ViewModel y se actualiza automáticamente. \\
\hline
\textbf{Comunicación} & Controller manipula el Modelo. View puede observar al Modelo directamente. & View delega los eventos a Presenter. Presenter actualiza a View a través de una interfaz. & View se suscribe a los flujos de datos de ViewModel. Los eventos de View ejecutan funciones en el ViewModel. \\
\hline
\textbf{Referencia a View} & Controller tiene una referencia directa a View. & Presenter tiene una referencia a View, pero a través de una interfaz abstracta. & ViewModel no tiene ninguna referencia a View. La comunicación es unidireccional (View observa a ViewModel). \\
\hline
\textbf{Testabilidad} & Difícil de probar de forma aislada debido al acoplamiento entre View y Controller. & Alta. Presenter es independiente de la UI de Android y se puede probar fácilmente con mocks. & Muy alta. ViewModel es completamente independiente de View, lo que facilita las pruebas unitarias. \\
\hline
\end{tabular}
\end{table}

El patrón MVVM se compone de tres componentes principales:

\begin{itemize}
  \item \textbf{Model:} Representa la lógica de negocio y los datos de la aplicación. En el launcher, Model incluye las clases que gestionan las tareas, hábitos, límites y preferencias del usuario.

  \item \textbf{View:} Es la interfaz de usuario que muestra los datos al usuario y recibe sus interacciones. Esta capa está implementada utilizando Jetpack Compose.

  \item \textbf{ViewModel:} Actúa como un intermediario entre Model y View. Contiene la lógica para preparar los datos que View necesita y maneja las interacciones del usuario. En el launcher, cada funcionalidad principal (tareas, hábitos, limites y preferencias) tiene su propio ViewModel.
\end{itemize}

\pagebreak

El flujo de datos en MVVM es unidireccional, lo que significa que View observa los cambios en el ViewModel y se actualiza automáticamente cuando los datos cambian. Esto se logra mediante el uso de bibliotecas reactivas como \textit{Flow}, que permite a View suscribirse a los cambios en los datos y reaccionar a ellos sin necesidad de acoplarse directamente a Model. La figura \ref{fig:arquitectura_mvvm} ilustra la relación entre estos componentes.

\begin{figure}[ht]
  \caption{Arquitectura MVVM.}
  \label{fig:arquitectura_mvvm}
  \includegraphics[width=\textwidth]{Figuras/arquitectura_mvvm.png}
  \centering
\end{figure}

\pagebreak

\subsection{Diseño.}

\subsubsection{Aplicaciones similares.}

En la fase inicial del proceso de diseño se llevó a cabo una revisión de aplicaciones existentes con enfoques similares al planteado, con el objetivo identificar funcionalidades relevantes y oportunidades de mejora que pudieran ser la base de las decisiones de diseño del proyecto. El análisis se concentró en tres launchers de código abierto que compartían la filosofía de minimalismo y funcionalidad enfocada en la productividad:

Hex Launcher \cite{HexLauncher} se destacó por su simplicidad visual y la organización de aplicaciones mediante categorías. Su interfaz limpia proporcionó indicadores sobre cómo reducir la sobrecarga visual sin comprometer la funcionalidad.

Launcher \cite{MinimalLauncher}, por su parte, se enfoca en atajos que sean útiles para acceder a las aplicaciones. Su diseño minimalista y la ausencia de elementos decorativos innecesarios sirvieron como inspiración para la creación de una interfaz que prioriza la funcionalidad sobre la estética.


Olauncher \cite{Olauncher}, contribuyó especialmente en demostrar la efectividad de un diseño extremadamente minimalista centrado en listas textuales en lugar de iconos gráficos para las aplicaciones. Esta aproximación confirmó la viabilidad de eliminar los elementos visuales tradicionalmente asociados con el reconocimiento instantáneo de aplicaciones, reemplazándolos por una interfaz que fomenta el uso consciente e intencional del dispositivo.

\subsubsection{Mockups y prototipo.}

Luego de haber identificado los elementos deseados en la interfaz, se empezaron a diseñar wireframes de baja fidelidad utilizando un tablero físico con marcadores borrables, explorando ideas y cambiando rápidamente los elementos que necesitaran ajustes. Durante estas sesiones, surgieron diversas aproximaciones para la organización de la información, la distribución de elementos en pantalla y los flujos de navegación entre diferentes funcionalidades. Desde el principio se tuvo en cuenta la necesidad de darle protagonismo a las funcionalidades de gestión de tareas y hábitos, ocupando gran parte de la pantalla de inicio con la intención de que el usuario siempre tuviera visible sus pendientes.

\begin{figure}[H]
  \caption{Wireframe en tablero.}
  \label{fig:wireframe_tablero}
  \includegraphics[width=0.5\textwidth]{Figuras/wireframe.jpg}
  \centering
\end{figure}


Una vez definidos los conceptos principales y validadas las ideas en el tablero físico, el proceso evolucionó hacia la creación de wireframes de alta fidelidad utilizando Figma. Esta transición permitió refinar los diseños con mayor precisión y establecer dimensiones específicas para los elementos de interfaz.

\begin{figure}[H]
  \caption{Wireframe de alta fidelidad.}
  \label{fig:wireframe_alta_fidelidad}
  \includegraphics[width=0.5\textwidth]{Figuras/wireframe_alta_fidelidad.png}
  \centering
\end{figure}

La fase final del proceso de diseño consistió en la transformación de los wireframes de alta fidelidad en mockups prototipados que incorporaran todos los aspectos visuales del diseño final. Esta etapa incluyó la definición de la paleta de colores, tipografías, espaciados, proporciones, íconos, acciones y animaciones. Se buscó crear una interfaz que no solo fuera visualmente agradable, sino que también facilitara la interacción del usuario con las funcionalidades clave del launcher, como la gestión de tareas, hábitos y límites. Los mockups y su prototipo se encuentran en \cite{Mockups}.

\begin{figure}[H]
  \centering
  \begin{minipage}{0.48\textwidth}
    \caption{Mockup: Pantalla principal.}
    \label{fig:mockup_pantalla_principal}
    \includegraphics[width=\textwidth]{Figuras/mockup_1.png}
    \centering
  \end{minipage}\hfill
  \begin{minipage}{0.48\textwidth}
    \caption{Mockup: Pomodoro.}
    \label{fig:mockup_pomodoro}
    \includegraphics[width=\textwidth]{Figuras/mockup_2.png}
    \centering
  \end{minipage}
\end{figure}

\begin{figure}[H]
  \centering
  \begin{minipage}{0.48\textwidth}
    \caption{Mockup: Menú de aplicaciones.}
    \label{fig:mockup_menu_aplicaciones}
    \includegraphics[width=\textwidth]{Figuras/mockup_3.png}
    \centering
  \end{minipage}\hfill
  \begin{minipage}{0.48\textwidth}
    \caption{Mockup: Configuraciones.}
    \label{fig:mockup_configuraciones}
    \includegraphics[width=\textwidth]{Figuras/mockup_4.png}
    \centering
  \end{minipage}
\end{figure}


%%%%%%%%%%%%%%%%%%%%%%%%%%%%%%Referencias%%%%%%%%%%%%%%%%%%%%%%%%%%%
\bibliography{Bibliografia}


\end{document}
