\section{Conclusiones.}
\label{sec:conclusiones}

\begin{itemize}
  \item Los estudiantes universitarios tienen una tendencia clara a usar su smartphone más tiempo del que deberían y en ocasiones inoportunas, reforzando la idea de implementar controles y hacer seguimiento al uso del smartphone para evitar distracciones, fomentando el cumplimiento de metas y deberes.
  \item El diseño minimalista, aunque efectivo para reducir distracciones, mostró la necesidad de ofrecer más opciones de personalización a los usuarios, tales como más tipografías, paletas de colores o poder añadir widgets; con el fin de mejorar el nivel de aceptación y adopción como aplicación de uso diario.
  \item La implementación de las tecnologías más recientes y recomendadas para iniciar nuevos proyectos en Android, concretamente Kotlin y Jetpack Compose, redujeron el tiempo de desarrollo y permitieron una integración completa que aprovecha el conjunto de características del lenguaje con su sintaxis simplificada.
  \item El patrón de arquitectura MVVM permitió que la aplicación escalara sin problemas a medida que se añadían nuevas funcionalidades, facilitando el mantenimiento, la comprensión y la reutilización del código gracias al principio de separación de responsabilidades.
  \item Las limitaciones técnicas del sistema operativo Android, como la actualización retardada de estadísticas de uso o el no poder cerrar las aplicaciones limitadas una vez cumplido el tiempo, impiden que este tipo de aplicaciones puedan trabajar acorde a lo que se espera e impactan negativamente la experiencia del usuario.
  \item Los resultados de las pruebas realizadas evidenciaron la satisfacción de los usuarios durante el periodo en el que interactuaron con la aplicación, expresando que esta fue de su agrado y que cumple con el objetivo de reducir las distracciones y ayudar a mejorar la productividad.
\end{itemize}